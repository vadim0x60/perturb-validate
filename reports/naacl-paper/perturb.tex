%
% File naacl2019.tex
%
%% Based on the style files for ACL 2018 and NAACL 2018, which were
%% Based on the style files for ACL-2015, with some improvements
%%  taken from the NAACL-2016 style
%% Based on the style files for ACL-2014, which were, in turn,
%% based on ACL-2013, ACL-2012, ACL-2011, ACL-2010, ACL-IJCNLP-2009,
%% EACL-2009, IJCNLP-2008...
%% Based on the style files for EACL 2006 by 
%%e.agirre@ehu.es or Sergi.Balari@uab.es
%% and that of ACL 08 by Joakim Nivre and Noah Smith

\documentclass[11pt,a4paper]{article}
\usepackage[hyperref]{naaclhlt2019}
\usepackage{times}
\usepackage{latexsym}

\usepackage{url}

%\aclfinalcopy % Uncomment this line for the final submission
%\def\aclpaperid{***} %  Enter the acl Paper ID here

%\setlength\titlebox{5cm}
% You can expand the titlebox if you need extra space
% to show all the authors. Please do not make the titlebox
% smaller than 5cm (the original size); we will check this
% in the camera-ready version and ask you to change it back.

\newcommand\BibTeX{B{\sc ib}\TeX}

\title{Perturbation learning for general-purpose text validation}

\author{Vadim Liventsev \\
  Center for Data-Intensive Science and Engineering \\
  Skolkovo Institute of Science and Technology\\
  3 Nobelya st., Moscow 121205 \\
  {\tt Vadim.Liventsev@skoltech.ru} \\\And
  Mariya Sheyanova \\
  School of Linguistics \\
  Higher School of Economics \\
  21/4 Staraya Basmannaya Ulitsa, Moscow 105066 \\
  {\tt mvsheyanova@edu.hse.ru} \\}

\date{}

\begin{document}
\maketitle
\begin{abstract}
  Language learners and generative models alike are often in need of text validation: checking how natural a certain sentence sounds within a given language or style.
  In this paper, we propose an approach to training a statistical validation model on a text corpus with no supervision.
  This is achieved by applying random perturbations to sentences from the corpus and training a recurrent neural network to discriminate between the original sentences and the perturbed ones.
  Choosing the right perturbation model, however, is far from trivial: the resulting validation model has to generalize beyond the specific perturbation we introduced and be able to recognize previously unseen kinds of deviations from the norm it learned from the corpus.
  We develop several perturbation models, demonstrate and compare their generalization ability.
\end{abstract}

\section{Background}
\label{sec:background}

Text validation is the problem of discriminating between text that belongs to a certaing domain (a language or a subdomain of a language, such as a certain author's style) from text that contains errors. 
Common applications of text validation include software that suggests improvements and error corrections for user-written text\footnote{for instance, Grammarly, \url{https://www.grammarly.com}} and as a quality control mechanism for generative models \cite{eval-genmodels}.

One way to develop a text validator is to manually implement a rule-based checker: an algorithm for text validation based on expert knowledge of the language at hand.
Early models, like \citet{easyenglish} and \citet{english-checker} for English, \citet{swedish-checker} for Swedish) worked like this: TODO. 
Modern solutions can make use of extensive research that has been done into \emph{constrained-based grammars}: there are frameworks like HPSG, many resource grammars like \cite{russian-grammar} for Russian and initiatives like DELPH-IN \footnote{Deep Linguistic Processing with HPSG Initiative, \url{http://www.delph-in.net/}}.
\citet{grammar-for-stylistics} shows how formal grammar can be used for stylistic validation. 
The obvious downside of using rule-based grammars is that they are language-specific and for every new language a new grammar has to be developed if not from scratch than with significant adaptation effort.

% Language modelling

% Using mistake data

% But here's why you need perturbation learning

\section{Methodology}
\label{sec:methodology}

We hypothesise that there exists a mechanism of applying random perturbations to sentences such that a discriminator trained to detect sentences that have been perturbed from intact ones can be used to detect mistakes more generally.
To that end, we introduce several \emph{perturbation models}.
For each of them, we train a binary classifier (\emph{validation model}), test its performance on a holdout validation dataset and then on the datasets used to train other \emph{validation models}.
Our hypothesis can be considered confirmed if a \emph{validation model} trained with \emph{perturbation model} correctly detect sentences modified with other \emph{perturbation models}.

\subsection{Perturbation models}

\begin{figure}
\end{figure}

\subsubsection{Word-order perturbations}

The first model we employ is \emph{random word flip}: a randomly selected word in the sentence is moved to a randomly selected location in the sentence.
All words and locations have equal probability to be selected.

\emph{Shuffle} perturbation means reordering the entire sentence according to a random permutation.

Note that neither of the models guarantees that the pertubed setence will be ungrammatical and, in fact, can leave the setence unchanged entirely.

\subsubsection{Word-form perturbations}

This kind of perturbation is performed using \texttt{pymorphy2} \cite{pymorphy2} and includes two types of transformations, based on morphological analysis and generation.

\begin{itemize}
    \item During \emph{random lemmatization}, each token in a sentence is either lemmatized with some probability (we use 50\% probability) or left as it is.
    \item \emph{Random inflection} is similar to \emph{random lemmatization}, but instead of replacing a token with its normal form, we take some other grammatical form of this word. For nouns, adjectives and personal pronouns, we randomly change case; for verbs, person is changed. Tokens with other parts of speech remain unchanged.
\end{itemize}

\subsubsection{Markov chain perturbations}

This type of perturbations differs from others in that instead of doing changes to an initially grammatical sentence, we train a generative n-gram language model to produce some ill-formed sentences. To create the language model, we used the \texttt{markovfy} \footnote{\url{https://github.com/jsvine/markovify}} implementation of Markov chain.

It is worth noting that not all of the sentences generated by markov chain are ungrammatical, but a significant part of them is, since the n-gram model cannot see further than n tokens into the past. In order to increase the number of ungrammatical sentences generated by the model we suppress any generated sentences that exactly overlap the original text by 50\% of the sentence's word count.

\subsection{Validation model}

Neural network-based approaches have the additional benefit that the validation function $f(s)$ ($s$ - sentence) is differentiable ($\frac{df}{ds}$ can be easily calculated) and thus can be used as perceptual loss \cite{perceptualloss} to train a generative neural network that outputs natural-sounding text.

\section{Experimental setup}
\label{sec:setup}

\section{Results}
\label{sec:results}
\bibliographystyle{acl_natbib}
\bibliography{../../references/refs}
\end{document}
